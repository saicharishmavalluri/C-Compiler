\documentclass{article}

\title{Welcome to My Compiler}
\author{Sai Charishma Valluri}

\begin{document}
\maketitle

\section{Introduction}
This document describes the purpose of each source file, the main data structures used, 
and a high-level overview of how the code works.

\section{Project - Part0}
This section deals with:
\begin{itemize}
  \item Initializing or getting ready the outline of the complier

  \item main.cpp: This source file takes the input of the command line and gives outputs 
  	or error messages based on the given command.
	
  \item a README.md file on how to build the compiler.

\end{itemize}

\section{Project - Part1}
This phase takes the modified source code as input files and breaks these syntaxes 
into a series of tokens by ignoring any whitespaces or comments in the source code.
\newline part1.cpp and lexer.l deals with the actual implementation of part1 project.
\begin{itemize}
  \item On a high level, part1.cpp takes the tokens from the lexer.l file and displays an appropriate message.
  \item main.cpp is same as in part0, except the intake of multiple input files 
  \item a README.md file on how to build the compiler.
  \item part1 requirements are implemented as below:
  \begin{itemize}
    \item Initially, taking all the token names from token table, and giving ASCII values to them in tokens.h file.
    \item As per token table, giving regular expressions for lexemes.
    \item used state condition to implement C-style comments, C++ style comments, \#include, \#define and \#ifdef.
    \item all the errors and warning messages are given to the standard error.
    \item NOTE: \#include, \#define, \#undef,\#ifdef, \#ifndef, \#else, \#endif were ignored and gives respective warning message.
  \end{itemize}
\end{itemize}
\section{Project - Part2}
This phase takes the modified source code as input files and displays the variable declarations, functions parameters and return types as the ouput.
\newline parts.cpp and part2.y deals with the actual implementation of part2 project.
\begin{itemize}
  \item On a high level, part2.y should take the tokens from the lexer.l file and applies specific grammar based on the structure of the input.
  \item main.cpp is same as in part0.
  \item a README.md file on how to build the compiler.
  \item part2 requirements are partially implemented as below:
  \begin{itemize}
    \item Initially, defined the grammar as per the section 2 of the given requirement.
    \item Based on the structure of the input, if there are no syntax errors, the program should display the details of functions and variables.
    \item used \%left and \%right for the associativity and precedence. 
    \item all the errors and warning messages are given to the standard error.
    \item NOTE: Only the grammar part is implemented for part2, with 48 shift/reduce conflicts.
  \end{itemize}
\end{itemize}
\end{document}
